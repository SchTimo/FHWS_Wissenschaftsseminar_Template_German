\section{Fazit}\label{sec:fazit}
Zum Schluss das Fazit oder die Schlußfolgerung mit Ausblick.
Soooo viel Text bis jetzt. 
Wie wäre es denn mal mit einem Bild?

Das könnt ihr ganz einfach mit einem Befehl machen. Doch zuerst solltet ihr das Bild (referenziert mit dem Befehl ref: \imageRef{fig:cat} auch in euer Verzeichnis einfügen. Dafür ist der Ordner fig gedacht. Das steht für figure! ;)
Im Anschluss einfach das Bild mit dem einfüge-Befehl referenzieren. Allerdings ist zu beachten, dass ihr es als Floating-Figure einfügt. Deswegen müsst ihr den Bereich mit begin einleiten:

\begin{figure}[tbp]
\centering % ZENTRIERT DAS BILD
\includegraphics[width=0.5\textwidth]{fig/mustacheCat} % DER BEFEHL IN ECKIGEN KLAMMERN GIBT DIE BREITE AN, DIE DAS BILD EINNEHMEN SOLL. HIER DIE HALBE TEXTBREITE.
\caption{Hier steht die Bildunterschrift\label{fig:cat}} % WENN IHR DAS ÜBER DEN BEFEHL INCLUDEGRAPHICS SETZT STEHT ES AUCH ÜBER DEM BILD.
\end{figure}