% %%%%%%%%%%%%%%%% HOW-TO %%%%%%%%%%%%%%%%
% DIESES FILE IST DIE MASTERDATEI. HIER MÜSST IHR NICHTS ANPASSEN, AUßER IHR WOLLT AN DER REIHENFOLGE ODER DER STRUKTUR DES PAPERS ETWAS ÄNDERN. ALLE INHALTE SOLLTEN IN DEN ENTSPRECHENDEN FILES ANGEPASST WERDEN (BUZZWORD: MODULARISIERUNG). ZUM VERÄNDERN DER REIHENFOLGE MÜSST IHR DIE EINZELNEN ABSCHNITTE EINFACH VERTAUSCHEN - WICHTIG: IHR DÜRFT NICHT DIE REIHENFOLGE DER PRÄAMBEL VERÄNDERN. AUCH IM CONTENT-BEREICH SOLLTE UM DIE STRUKTUR DES PAPERS ZU WAHREN ABSTRACT ALS ERSTES STEHEN GEFOLGT VON CATEGORIES UND KEYWORDS.

% DAS PROJEKT IST UNTERTEILT IN CONTENT UND DEFINITIONEN (ZU SEHEN IN DER PROJEKTSTRUKUT IM Z.B. FILEEXPLORER) IN BEREICH DEFINITIONEN FINDET IHR ALLES WAS DAS PAPER BESCHREIBT UND DEFINIERT - IN DIESEN FILES KÖNNT IHR EUREN RAHMEN VERÄNDERN. UM INHALTLICHE ANPASSUNGEN ZU MACHEN BRAUCHT IHR NUR (!) DIE FILES IM BEREICH CONTENT.

% UM KEINE FEHLER ZU ERZEUGEN MÜSSEN DIE BEREICHE \begin{document} und \end{document} AN IHREN POSITIONEN BELASSEN WERDEN -- DÜRFEN ALSO NICHT VERTAUSCHT WERDEN. ES GIBT HIER REGELN IN DER REIHENFOLGE DIE EINE ÜBERSETZUNG ERMÖGLICHEN -- WERDEN DIESE NICHT EINGEHALTEN FUNKTIONIERT DAS NICHT MEHR.

% %%%%%%%%%%%%%%%% HOW-TO %%%%%%%%%%%%%%%%

% %%%%%%%%%%%%%%%% PRÄAMBEL %%%%%%%%%%%%%%%%
% BEARBEITUNG DER EINZELNEN ABSCHNITTE IN DEN JEWEILIGEN FILES. 
% DIESE SOLLTEN LINKS (ABHÄNGIG VON DER VERWENDETEN BEARBEITUNGSSOFTWARE) IN DEN PROJEKTDATEIEN ZU SEHEN SEIN.

%============ DOCUMENTCLASS ================
% DIE DOCOMENTCLASS BESCHREIBT DEN TYPEN DES SCHRIFTSTÜCKES. ÜBLICH SIND EIGENTLICH BOOK (BUCH) ODER ARTICLE - HIERBEI HANDELT ES SICH ALLERDINGS UM EINE FHWS-EIGENKREATION FÜR DAS EIGENE FHWS-JOURNALE. DIE DEFINITION DIESES JOURNALS IST IN DER KLASSE fhwsjournale.cls BZW. fhwssingle.cls GEGEBEN. DA MUSS MAN EIGENTLICH NICHTS MEHR ANPASSEN - ALSO EINFACH SO NEHMEN WIE ES IST. DER UNTERSCHIED ZWISCHEN BEIDEN KLASSEN IST DAS JOURNALE FÜR EINE GEBUNDENE DUPLEXDRUCK VARIANTE GEBUAT WIRD. DIE SINGLE-VARIANTE BAUT FÜR EINSEITIGEN DRUCK - NICHT ZWINGEND GEBUNDEN. AKTUELL HABE ICH ALLERDINGS DIE JOURNALE-VARIANTE AUCH AUF EINSEITIGEN DRUCK EINGESTELLT -- DIE CLS DER FHWSSINGLE FÜHRT LEIDER ZU EINEM FEHLER -- DER ZWAR KEINE PROBLEME VERURSACHT ABER DA IST. 
\input{def/doc_class}
%============ DOCUMENTCLASS ================

%============ PAPERDATEN ================
% IN DIESEM FILE MÜSST (!) IHR DIE PAPERDATEN (TITLE, AUTHOR, …) ANGEBEN UND BEARBEITEN.

\fhwsYear{WS 2016/17}
       
\markboth{Your Name}{Wissenschaftsseminar}

\title{Hier bitte den spannenden Titel eintragen - ggf. den Untertitel hier}
            
\author{Your Name\\Wissenschaftsseminar\\Hochschule f{\"u}r angewandte Wissenschaften W{\"u}rzburg Schweinfurt}
        
%============ PAPERDATEN ================

%============ PACKAGES ================
% LATEX DEFINIERT SICH ÜBER PACKAGES, WELCHE EUCH MÖGLICHKEITEN DER VEREINFACHUNG GEBEN - VERGLEICHBAR ZU LIBRARYS FÜR JAVA. WANN IHR WELCHES PACKAGE BRAUCHT UND WIE MAN DIESE VERWENDET IST SITUATIONSABHÄNGIG UND MUSS ÜBER GOOGLE (O.Ä.) ERBRACHT WERDEN. 
\usepackage[ngerman]{babel}   % GIBT DIE SPRACHE AN NGERMAN FÜR DIE NEUE RECHTSCHREIBUNG
\usepackage[utf8]{inputenc}   % SORGT FÜR DIE EINBINDUNG VON UTF8 CODIERUNG UND SOMIT VON UMLAUTEN
\usepackage[T1]{fontenc}		  % SORGT FÜR DIE KORREKTE TRENNUNG VON WÖRTERN
\usepackage{csquotes}		  % ANFÜHRUNGSZEICHEN
\usepackage[hidelinks]{hyperref} % SORGT FÜR HYPERLINKS BEI REFERENZIERUNG ÜBER BEFEHL \ref
\usepackage{listings}		  % ERLAUBT CODELISTINGS
\usepackage{caption}		  % OPTIMIERT CAPTIONS
\usepackage{nameref}
\usepackage{hyperref}
%============ PACKAGES ================

%============ DEFINITIONS ================
% IN DIESEM BEREICH WERDEN DIE DEFINITIONEN INTEGRIERT.
\newcommand{\imageRef}[1]{Abbildung: \ref{#1}}
%============ DEFINITIONS ================

% %%%%%%%%%%%%%%%% PRÄAMBEL %%%%%%%%%%%%%%%%                      


% %%%%%%%%%%%%%%%% CONTENT %%%%%%%%%%%%%%%%   
% HIER HABE ICH IN DER CLS-FILE DIE NAMEN FÜR CATEGORY, TERMS, KEYWORDS INS DEUTSCHE ÜBERSETZT. WENN DAS NICHT GEWÜNSCHT IST EINFACH IN DER CLS WIEDER ÄNDERN. ICH HABE DAS ÜBER KOMMENTARE GESTEUERT. IST EIGENTLICH GANZ EINFACH. EINFACH IN DIE CLS GEHEN UND NACH DEM WORD KEYWORDS SUCHEN. DANN FINDET IHR ES RECHT SCHNELL.         

%============ ABSTRACT ================
% ZUSAMMENFASSUNG EURER ARBEIT

\begin{abstract}
Abstract bla bla  
\end{abstract}
%============ ABSTRACT ================

%============ CATEGEORY ================
% WELCHE KATEGORIE HAT DAS PAPER?
% ES GIBT:
%		G = General
%		H = Hardware
%		IT= IT-Organisation
%		S = Software
%		IS= Information Systems
%		Sub=Category is "free"
% ES GIBT:
%		G = General
%		H = Hardware
%		IT= IT-Organisation
%		S = Software
%		IS= Information Systems
%		Sub=Category is "free"

\category{G}{General}{Information Systems}

%============ CATEGEORY ================            

%============ TERMS ================                      
% HAUPT-KEYWORDS. HIER STEHEN EIN PAAR BUZZWORDS ZUR THEMATIK
\input{content/terms}
%============ TERMS ================   

%============ KEYWORDS ================   
% HIER STEHEN ERWEITERTE KEYWORDS.                 
% KEYWORDS DER ARBEIT BZW. DES THEMAS - EINFACH MIT KOMMA TRENNEN
\keywords{RETE, Parallelisierung, ...}  
%============ KEYWORDS ================        


% #+#+#+#+# START MAINCONTENT +#+#+#+#+#+#
% HIER BEGINNT DER MAINCONTENT 
%-------- DO NOT TOUCH THIS AREA --------------
\begin{document}
%-------- DO NOT TOUCH THIS AREA --------------

%============ FOOTER ================  
% HIER WIRD DER TEXT IN DER FUßZEILE BEARBEITET. DAS LIZENZBLABLA DEFINIERT SICH ÜBER DIE CLS-DATEI. SOLLTE ABER NICHT VERÄNDERT WERDEN. AKTUELL HABE ICH ALLES AUSKOMMENTIERT. WENN IHR DAS WIEDER EINFÜGEN WOLLT EINFACH IN DER FILE DIE PROZENTSYMBOLE ENTFERNEN.
%\begin{bottomstuff} 
%Was man halt so in eine Fußzeile schreibt…
%\end{bottomstuff}

%============ FOOTER ================  

%-------- DO NOT TOUCH THIS AREA --------------
% HIER WIRD DER TITEL ERZEUGT - NICHT VERSCHIEBEN ODER VERÄNDERN ODER ENTFERNEN
\maketitle
%-------- DO NOT TOUCH THIS AREA --------------

%============ SECTION - AREA ================
% IN DIESEM BEREICH STEHEN DIE EINZELNEN KAPITEL

%-------- KAPTIEL 1 --------------
% HIER SOLLTE DIE EINLEITUNG/HINFÜHRUNG/PROBLEMERÖRTERUNG STEHEN
\section{Einleitung}\label{sec:intro}
Hier steht die Einleitung...
%-------- KAPTIEL 1 --------------

%-------- KAPTIEL 2 --------------
% WELCHE ANDEREN ANSÄTZE WURDEN IN DER THEMATIK BEREITS DURCH ANDERE TEAMS ODER PERSONEN UNTERNOMMEN - WAS GIBT ES FÜR ALTERNATIVE ANSÄTZE.
\section{Anverwandte Arbeiten}\label{sec:relatedWork}
In diesem Bereich solltet ihr Bezug auf verwandte Themen und Arbeiten nehmen.
Es bietet sich natürlich auch an diese zu referenzieren - bringt quellen ;)

Referenzen auf eine Quelle macht man wie folgt:
\begin{enumerate}
	\item Hinzufügen der BibTex-Literatur-Referenz in die bib-File (fhwsbib.bib). 
	\item Buch zitieren mit dem Befehl cite. Z.\,B. bei Harry Potter = \cite{Rowling2015}
\end{enumerate}


%-------- KAPTIEL 2 --------------

%-------- KAPTIEL 3 --------------
% IN KAPITEL 3 SOLLTE MAN SICH MIT DEN MATERIALIEN UND METHODEN AUSEINANDERSETZEN - WIE HABE ICH MEINE LÖSUNG GEFUNDEN - WISSENSCHAFTLICHE METHODEN... 
\section{Methoden}\label{sec:meth}
In diesem Bereich stehen eure wissenschaftlichen Methoden. 

Wofür ist eigentlich das Label? Das Label dient der Referenzierung einzelner Sections (Abschnitte). Diese könnt ihr mit dem Befehl \emph{ref} referenzieren. Sieht dann so aus: hier referenziere ich \ref{sec:intro} => das Intro.
%-------- KAPTIEL 3 --------------

%-------- KAPTIEL 4 --------------
% IN KAPTEL 4 SOLLTET IHR EURE LÖSUNG VORSTELLEN. QUASI DER HAUPTTEIL.
\section{Hauptteil}\label{sec:main}
Hier steht eure Lösung -- also quasi der Hauptteil.

Der Hauptteil ist sooo wichtig. den sollte man doch mal mit fettgeschriebenen Worten schmücken. Wenn ihr das machen wollt, dann müsst ihr den Befehl \emph{textbf} verwenden. Steht für Text-Bold-Font (glaube ich :D ) -- das sieht dann so aus: \textbf{ich bin fett geschrieben.}.

Kursiv geht übrigens auch -- es bietet sich allerdings an den Befehl \emph{emph} (emphasize) dafür zu verwenden -- dieser verhält sich im Layoutverhalten besser als der übliche Befehl \emph{textit}. Das sähe dann so aus emph: \emph{emphasize} vs. italic: \textit{italic-kursiv}.


%-------- KAPTIEL 4 --------------

%-------- KAPTIEL 5 --------------
% KAPITEL 5 BEHANDELT EINE ART EVALUATION. HIERBEI GILT ES EURE ARBEIT IN RELATION ZU SETZEN ZU ANDEREN ANSÄTZEN ODER GENERELL ZU EVALUIEREN. ES SOLLTE AUCH SEHR KRITISCH DISKUTIERT WERDEN.
\section{Evaluation}\label{sec:evaluation}
Hier sollt ihr eure Lösung evaluieren und diskutieren.

Vllt mag man hier ja mal ein paar Dinge aufzählen. Das könnt ihr ganz einfach mit dem Befehl itemize für Bulletpointlisten und dem Befehl enumerate für Aufzählungen. Wichtig ist hierbei jedoch zu beachten, dass es sich hierbei um einen Bereich handelt, der mit begin eingeleitet und mit end beendet wird und kein command ist.
\begin{itemize}
	\item Ein bulletpoint
	\item noch einer
	\item uuuuhhhh ein dritter
	\item jetzt wirds lanweilig...
\end{itemize}
Im Vergleich dazu die Aufzählung:
\begin{enumerate}
	\item Nummer eins
	\item zwei
	\item drei
	\item hey du kannst zählen
	\item sieben
	\item ach doch nicht...
	\item ...sechs vergessen
\end{enumerate}
%-------- KAPTIEL 5 --------------

%-------- KAPTIEL 6 --------------
% KAPITEL 6 BILDET DEN SCHLUSS - HIER SOLL EIN FAZIT/SCHLUßFOLGERUNG/AUSBLICK BESCHRIEBEN WERDEN.
\section{Fazit}\label{sec:fazit}
Zum Schluss das Fazit oder die Schlußfolgerung mit Ausblick.
Soooo viel Text bis jetzt. 
Wie wäre es denn mal mit einem Bild?

Das könnt ihr ganz einfach mit einem Befehl machen. Doch zuerst solltet ihr das Bild (referenziert mit dem Befehl ref: \imageRef{fig:cat} auch in euer Verzeichnis einfügen. Dafür ist der Ordner fig gedacht. Das steht für figure! ;)
Im Anschluss einfach das Bild mit dem einfüge-Befehl referenzieren. Allerdings ist zu beachten, dass ihr es als Floating-Figure einfügt. Deswegen müsst ihr den Bereich mit begin einleiten:

\begin{figure}[tbp]
\centering % ZENTRIERT DAS BILD
\includegraphics[width=0.5\textwidth]{fig/mustacheCat} % DER BEFEHL IN ECKIGEN KLAMMERN GIBT DIE BREITE AN, DIE DAS BILD EINNEHMEN SOLL. HIER DIE HALBE TEXTBREITE.
\caption{Hier steht die Bildunterschrift\label{fig:cat}} % WENN IHR DAS ÜBER DEN BEFEHL INCLUDEGRAPHICS SETZT STEHT ES AUCH ÜBER DEM BILD.
\end{figure}
%-------- KAPTIEL 6 --------------

%============ SECTION - AREA ================

% HIER ENDET DER MAINCONTENT 
% #+#+#+#+# ENDE MAINCONTENT +#+#+#+#+#+#

% %%%%%%%%%%%%%%%% CONTENT %%%%%%%%%%%%%%%%                        
            
% %%%%%%%%%%%%%%%% BIBLIOGRAPHIE %%%%%%%%%%%%%%%%  
% DIE BIBLIOGRAPHY SORGT FÜR EUER LITERATURVERZEICHNIS. DAS STEHT DANN UNTER REFERENCES.               

%-------- DO NOT TOUCH THIS AREA --------------
% IM COMMAND \bibliography GEBT IHR DIE FILE (ENDUNG .bib) EIN, IN WELCHER EURE BITEX ABSCHNITTE STEHEN - DIESE KÖNNT IRH ÜBRIGENS ENTWEDER DIREKT VON IEEE ODER ACM ODER SPRINGER LINK BEZIEHEN. ALTERNATIV GIBT ES AUCH DIE SEITE: http://www.literatur-generator.de/ DIE BIBTEX BEZIEHEN.

% IN DEM STYLE WÄHLT IHR DIE ART DER ZITIERUNG. ÜBLICHE STYLES SIND AUF DIESER SEITE ZU FINDEN: https://de.sharelatex.com/learn/Bibtex_bibliography_styles ALLERDINGS GIBT ES AUCH DEN STYLE DER in fhwsjournale.bst DEFINIERT IST. DIESER IST HIER VERWENDET.
\bibliography{fhwsbib}
\bibliographystyle{fhwsjournale}
%-------- DO NOT TOUCH THIS AREA --------------
              
% %%%%%%%%%%%%%%%% BIBLIOGRAPHIE %%%%%%%%%%%%%%%%    

% %%%%%%%%%%%%%%%% SONSTIGES %%%%%%%%%%%%%%%%                            
% AKTUELL STEHT HIER NUR DER RECEIVED PART. DIESER BESCHREIBT BEI EINEM PAPER DEN ABSTAND ZWISCHEN ERSTEINREICHUNG AUF EINER KONFERENZ UND DEM MONAT/JAHR DER ANNAME => JE GERINGER DIESER ABSTAND UM SO AKTUELLER UND GGF. BESSER IST DAS PAPER

\begin{received} 
Received Month Year; 
revised Month Year; accepted Month Year 
\end{received} 


%-------- DO NOT TOUCH THIS AREA --------------
\end{document}
%-------- DO NOT TOUCH THIS AREA --------------
% %%%%%%%%%%%%%%%% SONSTIGES %%%%%%%%%%%%%%%%                                        
            
            

            
